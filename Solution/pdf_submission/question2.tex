\section{Mistake Bound Model of Learning}\label{sec:q2}

\begin{enumerate}
    \item~
    
    
	\begin{enumerate}
		\item
			Since $1 \leq l \leq 80$, the size of concept class $|C| = 80$	
		
		\item
			The prediciton made by hypothesis of length $l$ on inputs $x_1^t$ and $x_2^t$ can be represented by the equation $$ sgn(2l - |x_1^t| - |x_2^t|) $$ 
			We make a mistake when this prediction does not match $sgn(y^t)$ of the label $y^t$. Hence we can determine whether mistake has been made using the following inequality:					
			\begin{equation}
				\boxed{				
					y^t(2l - |x_1^t| - |x_2^t|) < 0	
				}	
			\end{equation}
	\end{enumerate}	    
    
    
    
    
    

    \item~[20 points] Recall from class that the Halving algorithm assumes that there is the true (hidden) function is in the concept class $\mathcal{C}$ with $N$ elements and tries to find it. In this setting, we know the number of mistakes made by the algorithm is $O(\log N)$.
        Another way to think about this setting is that we are trying to predict with expert advice. That is, we have a pool of $N$ so called experts, only one of whom is perfect. As the halving algorithm proceeds, it cuts down this pool by at least half each time a mistake is made.

        Suppose, instead of one perfect expert, we have M perfect experts in our pool. Show that the mistake bound of the same Halving algorithm in this case is $O(\log\frac{N}{M})$.
        (Hint: To show this, consider the stopping condition of the algorithm. At what point, will the algorithm stop making mistakes?)
\end{enumerate}

%%% Local Variables:
%%% mode: latex
%%% TeX-master: "hw"
%%% End:
